\documentclass[10pt]{article}

\usepackage{amsmath}
\usepackage{textcomp}
\usepackage[top=0.8in, bottom=0.8in, left=0.8in, right=0.8in]{geometry}
% Add other packages here %
\usepackage{algorithm}
\usepackage{algpseudocode}
\usepackage{changepage}
\usepackage{float}
\usepackage[pdftex]{graphicx}
\usepackage{caption}
\usepackage{subcaption}
\usepackage[font=scriptsize]{subcaption}
\usepackage[font=scriptsize]{caption}

% Put your group number and names in the author field %
\title{\bf Excercise 3\\ Implementing a deliberative Agent}
\author{Group \textnumero 23: Jean-Thomas FURRER, Emily HENTGEN}

% N.B.: The report should not be longer than 3 pages %

\begin{document}
\maketitle

\section{Model Description}

\subsection{Intermediate States}
% Describe the state representation %
In our model, a state is characterized by the following six attributes:
\begin{itemize}
\itemsep 0mm
\item[\textendash]int[] \textbf{tasksStatus}: keeps track of whether each task has not been picked up yet, has been picked up but not delivered, or has been delivered
\item[\textendash]int \textbf{taskIndex}: the index of the task being currently handled in this state
\item[\textendash]City \textbf{departure}: the current city where the agent's vehicle is
\item[\textendash]double \textbf{cost}: the accumulated cost in this state
\item[\textendash]double \textbf{charge}: the accumulated charge in this state
\item[\textendash]State \textbf{previous}: the previous state the agent's vehicle was in
\end{itemize}

Each state additionally keeps a reference to the list of tasks that have to be delivered.

\subsection{Goal State}
% Describe the goal state %
A goal (or final) state is a state where all tasks have been delivered. For a given state, this can be checked using the \textit{tasksStatus} attribute.

\subsection{Actions}
% Describe the possible actions/transitions in your model %
There are three possible actions in our model: moving from one city to a neighbouring city, picking up a task and delivering a task.

\section{Implementation}

\subsection{BFS}
% Details of the BFS implementation %
The main issue with the BFS algorithm is that the number of states grows exponentially with the number of tasks.
The implementation must hence create states occupying few memory and each state visit must be as fast as possible.
To this end, we use a \textit{LinkedList} for storing the set of all states that are yet to be visited and \textit{HashSet}s for storing the successors of the current state as well as the set of all the already visited states.
In the previous \textit{State} definition, we further use an array of \textit{int} for keeping track of each task status: it occupies few memory, and updating it induces less overhead than \textit{List}s or \textit{Set}s for instance storing the yet to be picked up/picked up but not yet delivered tasks.
Following the same idea, instead of storing the complete list of actions that lead to the state, we store a reference to the previous state, which is less memory consuming.

\begin{algorithm}[t]
\caption{BFS}
\begin{algorithmic}[1]
\Statex \textsc{Input}
\Statex \hspace{\algorithmicindent} Vehicle \textit{vehicle} \Comment{the agent's vehicle}
\Statex \hspace{\algorithmicindent} TaskSet \textit{tasks} \Comment{the set of tasks to deliver}
\Statex \textsc{Output}
\Statex \hspace{\algorithmicindent} Plan \textit{plan} \Comment{the plan found by the BFS algorithm}

\Statex	  
\State $Q \gets \{initialState\}$, $S \gets \{\}$, $loopCheck \gets \{\}$
\State $minimumCost \gets \infty$, $finalState \gets null$ 
\While{$Q$ is not empty}
	\State $currentState \gets Q.pop$
	\If{$currentStat$ is a final state}
		\If{$currentState.cost < minimumCost$}
			\State $minimumCost \gets currentState.cost$
			\State $finalState \gets currentState$
		\EndIf
	\EndIf
	
	\If{$loopCheck$ does not contain $currentState$}	
		\State $loopCheck \gets loopCheck \cup \{currentState\}$
		\State $S \gets currentState.successors$
		\State $Q \gets Q \cup S$
	\EndIf
\EndWhile
\Statex $plan \gets buildPlan(finalState, plan, tasks)$
\State \Return $plan$
\end{algorithmic}
\end{algorithm}

%\begin{algorithm}[t]
%\caption{buildPlan}
%\begin{algorithmic}[1]
%\Statex \textsc{Input}
%\Statex \hspace{\algorithmicindent} State \textit{state} \Comment{the state from which to recursively start building the plan}
%\Statex \hspace{\algorithmicindent} List\textless Task\textgreater\ \textit{tasks} \Comment{the list of tasks}
%\Statex \hspace{\algorithmicindent} Plan \textit{plan} \Comment{the plan being recursively built}
%\Statex \textsc{Output}
%\Statex \hspace{\algorithmicindent} Plan \textit{plan} \Comment{the plan built starting from the given state}
%
%\Statex
%%\State $plan \gets Plan(vehicle.getCurrentCity)$
%\State $previousState \gets state.previous$	   
%\If{$previousState$ is not $null$}
%	\State buildPlan($previousState, plan, tasks$)
%	\State $i \gets state.taskIndex$
%	\For{$city$ in $previousState.departure.pathTo(state.departure)$}
%		\State $plan.appendMove(city)$
%	\EndFor	
%	\If{$state.taskStatus[i]$ is pickedUp}
%		\State $plan.appendPickup(tasks[i])$
%	\ElsIf{$state.taskStatus[i]$ is delivered}
%		\State $plan.appendDelivery(tasks[i])$
%	\EndIf	
%\EndIf
%\State \Return $plan$
%\end{algorithmic}
%\end{algorithm}



\subsection{A*}
% Details of the A* implementation %
The A* algorithm is similar to the BFS algorithm. The main difference is that the queue of states is reordered as soon as new (successors) states are added to it. Moreover, once a final state is reached, the algorithm does not further visit other states, but terminates and uses this final state for building a plan.

\subsection{Heuristic Function}
% Details of the heuristic functions: main idea, optimality, admissibility %
The main idea behind our heuristic function is that we want to minimize the final cost. In the BFS algorithm, once we reached all the final states, we keep only one which is the one having the smallest cost. Hence the simplest heuristic function can simply be to compare the cost of reaching two states. In that way, we can be sure that when reaching the very first final state, this is also the optimal one and we can stop the algorithm.

\section{Results}

\subsection{Experiment 1: BFS and A* Comparison}
% Compare the two algorithms in terms of: optimality, efficiency, limitations %
% Report the number of tasks for which you can build a plan in less than one minute 

\subsubsection{Setting}
% Describe the settings of your experiment: topology, task configuration, etc. %
Topology: Switzerland map\\
Tasks configuration: tasks uniformly distributed\\
Vehicle: default configuration

\subsubsection{Observations}
% Describe the experimental results and the conclusions you inferred from these results %
The BFS algorithm runs out of memory when the number of tasks to deliver exceeds 8; the A* algorithm can go up to 9 tasks (in less than a minute) before it throws an \textit{OutOMemoryError} as well, or times out.
More detailed performance results for both algorithms with 6 up to 10 tasks and a seed of 23456 are shown in the Table \ref{table_seed23456}.\\

\begin{adjustwidth}{0cm}{}
\begin{tabular}{|l|lllll|llllll|}
\hline
Number of tasks & & 6 & 7 & 8 & 9 & & 6 & 7 & 8 & 9 & 10\\
\hline
Minimum cost & & 6900 & 8050 & 8550 & - & & 6900 & 8050 & 8550 & 8600 & -\\
Execution time (ms) & BFS & 113 & 1155 & 9723 & - & A* & 102 & 230 & 2146 & 18147 & -\\
Number of states & & 49 993 & 435 391 & 2 517 570 & - & & 22 097 & 222 660 & 1 351 656 & 6 420 867 & -\\
\hline
\end{tabular}
\captionof{table}{Performance of BFS and A* for a seed equal to 23456}
\label{table_seed23456}
\end{adjustwidth}
\vspace{4mm}

\noindent
By changing the seed to 65432 for instance, the number of tasks for which the agent can compute a plan increases: the BFS agent can now handle up to 9 tasks, and the A* agent up to 10. We notice the cost for 9 and 10 tasks is the same with the A* algorithm: as the number of tasks increases, the probability that the agent will have to pick up or deliver a task to a city it previously only passed through increases as well. If its capacity allows it, it can thus design a plan which includes the additional task with a cost no higher than before (Table \ref{table_seed65432}).\\
\begin{adjustwidth}{0cm}{}
\begin{tabular}{|l|lllll|llllll|}
\hline
Number of tasks & & 7 & 8 & 9 & 10 & & 7 & 8 & 9 & 10 & 11\\
\hline
Minimum cost & & 5700 & 6600 & 7150 & - & & 5700 & 6600 & 7150 & 7150 & -\\
Execution time (ms) & BFS & 260 & 3122 & 31 627 & - & A* & 68 & 472 & 4501 & 52 153 & -\\
Number of states & & 115 157 & 991 288 & 5 307 498 & - & & 46 464 & 437 334 & 2 488 459 & 10 144 872 & -\\
\hline
\end{tabular}
\captionof{table}{Performance of BFS and A* for a seed equal to 65432}
\label{table_seed65432}
\end{adjustwidth}
\vspace{4mm}

\noindent
For this seed, BFS is able to build a plan in less than one minute for up to 9 tasks, and A* for up to 10 tasks. This number can vary depending on the initial setting and the seed though. It can also slightly change across different simulations, although the initial configuration is the same.\\

\noindent
Compared to the BFS algorithm, the A* algorithm can only design a plan with one additional task before tt eventually times out or throws an \textit{OutOfMemoryError}.
However, in overall, since it visits less states, the A* algorithm also takes less time to compute an optimal plan the BFS algorithm finds by visiting all the possible states.

\subsection{Experiment 2: Multi-agent Experiments}
% Observations in multi-agent experiments %

\subsubsection{Setting}
% Describe the settings of your experiment: topology, task configuration, etc. %
We use the default configuration, with the map of Switzerland, and a seed of 23456. In this setting, there are up to three agents competing: the first agent starts from Lausanne, the second from Z\"urich and the third from Bern.

\subsubsection{Observations}
% Describe the experimental results and the conclusions you inferred from these results %
As the number of agents increases, their average reward per km decreases, since they are not coordinated.
Moreover, one plan recomputation induces a cascade of other plan recomputations: after the agent starting from Z\"urich performs in the beginning one plan recomputation, the agent starting from Lausanne ends up doing four plan recomputations. Each new plan involves a task the agent from Z\"urich happens to just have picked up, so in the end, the agent from Lausanne delivers only the task it picked up before the conflict occurred.
With fewer tasks, four for instance, one agent may end up not delivering any tasks, as shown in Figure 1 (d).


\begin{figure}[h!]
\centering
\begin{subfigure}[t]{0.24\textwidth}
\captionsetup{width=1.0\textwidth}
\includegraphics[trim={0cm 0cm 0cm 0cm},clip, scale=0.3]{images/"multiagents1_tasks7_seed23456".png}
\caption{One deliberative agent using the BFS algorithm, 7 tasks, seed = 23456}
\end{subfigure}
\hfill
\begin{subfigure}[t]{0.24\textwidth}
\captionsetup{width=1.0\textwidth}
\includegraphics[trim={0cm 0cm 0cm 0cm},clip, scale=0.3]{images/"multiagents2_tasks7_seed23456".png}
\caption{Two deliverative agents using the BFS algorithm, 7 tasks, seed = 23456}
\end{subfigure}
\hfill
\begin{subfigure}[t]{0.24\textwidth}
\captionsetup{width=1.0\textwidth}
\includegraphics[trim={0cm 0cm 0cm 0cm},clip, scale=0.3]{images/"multiagents3_tasks7_seed23456".png}
\caption{Three deliberative agents using the BFS algorithm, 7 tasks, seed = 23456}
\end{subfigure}
\hfill
\begin{subfigure}[t]{0.24\textwidth}
\captionsetup{width=1.0\textwidth}
\includegraphics[trim={0cm 0cm 0cm 0cm},clip, scale=0.3]{images/"multiagents3_tasks4_seed23456".png}
\caption{Three deliberative agents using the BFS algorithm, 4 tasks, seed = 23456}
\end{subfigure}
\caption{Competing deliberative agents using the BFS algorithm}
\label{discount factor}
\end{figure}

\noindent
Theoretically, an agent using the A* algorithm should outperform an agent using the BFS algorithm, since it could begin picking up tasks sooner, while the other agent would still be computing its optimal plan.
The platform however does not allow us to observe this phenomenon, because the agents start moving when all agents are done computing their plan.



\end{document}